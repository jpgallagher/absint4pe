\section{Transformation of the big-step semantics for the call-by-value $\lambda$-calculus}\label{lambda-ex}

\begin{figure}
\[
\begin{array}{l|l}
 \dfrac{ } 
{\langle \VAL{v}, \rho \rangle \Longrightarrow v}


&~~~
 \dfrac{ } 
{\langle \VAR{x}, \rho \rangle \Longrightarrow v}

 ~~~~~~~\text{if }  \rho x = v\\
 \\
\dfrac{ } 
{\langle \LAM{x}{e}, \rho \rangle \Longrightarrow \CLO{x}{e}{\rho}}~~~
&~~~
\dfrac{\langle e_1, \rho \rangle \Longrightarrow \CLO{x}{e}{\rho'} ~~~\langle e_2, \rho \rangle \Longrightarrow v_2~~~
\langle e, \rho'[x/v_2] \rangle \Longrightarrow v} 
{\langle \APP{e_1}{e_2}, \rho \rangle \Longrightarrow v}
 \\
 \\
 \end{array}
\]
\caption{Big-step semantics for the call-by-value $\lambda$-calculus.}\label{big-step-trans-lambda}
\end{figure}


\begin{figure}
\[
\begin{array}{l}

\dfrac{\langle e_1, \rho \rangle \Longrightarrow \CLO{x}{e}{\rho'} ~~~\langle \APPP {\CLO{x}{e}{\rho'}}{e_2}, \rho \rangle \Longrightarrow v} 
{\langle \APP{e_1}{e_2}, \rho \rangle \Longrightarrow v}
\\
\\
\dfrac{\langle e_2, \rho \rangle \Longrightarrow v_2~~~
\langle e, \rho'[x/v_2] \rangle \Longrightarrow v} 
{\langle \APPP{ \CLO{x}{e}{\rho'}}{e_2}, \rho \rangle \Longrightarrow v}
 \\
 \\
 \end{array}
\]
\caption{Modified rule for $\keyword{app}$ introducing new constructor $\keyword{app2}$.}\label{app2-trans}
\end{figure}

\begin{figure}
\begin{tabular}{l}
\begin{lstlisting}
smallStep(var(C),A,B,[]) :- eval__3(A,C,B).
smallStep(val(A),_,A,[]).
smallStep(lam(B,C),A,clo(B,C,A),[]).
smallStep(app(C,D),A,B,[bigstep(app2(clo(E,F,G),D),A,B)]) :-
    leaf__4(bigstep(C,A,clo(E,F,G))),
    smallStep(C,A,clo(E,F,G),[]).
smallStep(app(C,D),A,B,[bigstep(app(E,D),A,B)]) :-
    nonLeaf__5(bigstep(C,A,clo(F,G,H))),
    smallStep(C,A,clo(F,G,H),[bigstep(E,A,clo(F,G,H))]).
smallStep(app2(clo(D,E,F),C),A,B,[bigstep(E,G,B)]) :-
    leaf__4(bigstep(C,A,H)),
    eval__6(D,H,F,G),
    smallStep(C,A,H,[]).
smallStep(app2(clo(D,E,F),C),A,B,[bigstep(app2(clo(D,E,F),H),G,B)]) :-
    nonLeaf__5(bigstep(C,A,I)),
    smallStep(C,A,I,[bigstep(H,G,I)]).
\end{lstlisting}
\end{tabular}
\caption{Small step clauses for $\lambda$-calculus, in the specialised interpreter.}\label{small-step-lambda}
\end{figure}

The modified rules are obtained from the original rule for $\keyword{app}$, which has three premises.  
The second and third premises are combined into a single big step, and the constructor $\keyword{app2}$
is introduced to collect the syntax terms in the two premises.  The premises of the rule for $\keyword{app2}$ 
is the same as the second and third premises of the original $\keyword{app}$ rule.

Such a transformation is applied whenever a rule has more than two big-step premises.  The result is
a set of rules in which no rule has more than two premises.  The transformation is carried out here on
the big-step semantic rules, but it could be done in the interpreter, with a new constructor (such as $\keyword{app2}$)
introduced whenever a rule with more than two premises is encountered.

\begin{figure}
\[
\begin{array}{l|l}
 \dfrac{ } 
{\langle \VAL{v}, \rho \rangle \Longrightarrow v}
 ~~~&~~~
 \dfrac{\langle e_1, \rho \rangle \Rightarrow \CLO{x}{e}{\rho'} } 
{\langle \APP{e_1}{e_2}, \rho \rangle \Rightarrow \langle \APPP{\CLO{x}{e}{\rho'}}{e_2},\rho\rangle}
 \\
 \\
  \dfrac{ } 
{\langle \VAR{x}, \rho \rangle \Rightarrow v}

 ~~~\text{if }  \rho x = v
&~~~
 \dfrac{\langle e_1, \rho \rangle \Rightarrow \langle e_3, \rho \rangle } 
{\langle \APP{e_1}{e_2}, \rho \rangle \Rightarrow \langle \APP{e_3}{e_2},\rho\rangle}

\\
 \\
 
\dfrac{ } 
{\langle \LAM{x}{e}, \rho \rangle \Rightarrow \CLO{x}{e}{\rho}}~~~
&~~~
\dfrac{\langle e_2, \rho \rangle \Rightarrow v_2}
{\langle \APPP{\CLO{x}{e}{\rho'}}{e_2},\rho\rangle \Rightarrow \langle e, \rho'[x/v_2] \rangle}
\\
\\
&~~~
\dfrac{\langle e_2, \rho \rangle \Rightarrow \langle e_3, \rho'' \rangle}
{\langle \APPP{\CLO{x}{e}{\rho'}}{e_2},\rho\rangle \Rightarrow \APPP{\CLO{x}{e}{\rho'}}{e_3},\rho''\rangle}

 \\
 \\
 


~~~&~~~
 \end{array}
\]
\caption{Extracted small-step rules for the $\lambda$-calculus.}\label{small-step-rules-lambda}
\end{figure}
