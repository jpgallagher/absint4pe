\begin{abstract}

The operational semantics of a programming language is specified as
transition rules, usually following one of two styles: natural semantics
(big-step) or structural operational semantics (small-step). Each style
has its advantages and disadvantages, so it can be useful to produce
both semantic forms for a language. Previous work has shown that
big-step semantics can be transformed to small-step semantics. This is
also the goal of our work, but our main contribution  is to show that this can be done
by specialisation of an interpreter that imposes a small-step execution
on big-step transition rules.  This is more
direct and transparent than previous transformation methods, and allows
variations, such as the size of the small steps, as minor modifications.
It also exploits standard features of partial evaluation to handle
low-level manipulation of the transitions. 
%We have previously shown that
%a big-step Horn clause representation of an imperative program (as
%opposed to the semantic rules themselves) can be transformed to a
%small-step representation using a linearising interpreter.  The idea in
%this work is similar, but special attention is given to encoding the
%form of the steps required for the small-step transitions.

\end{abstract}